
\section{Introduction}

In recent years, advancements in measurement technologies have enabled the collection of vast amounts of multi-omics data, ranging from genomic and epigenomic profiles to transcriptomic and proteomic measurements. These datasets contain valuable information about the molecular signatures underlying complex diseases, offering opportunities for understanding disease mechanisms, identifying potential therapeutic targets, and developing personalized treatment strategies \cite{konig2017precision}.

Due to the paradigm shift towards personalized medicine \cite{aronson2015building}, there has been growing interest in leveraging multi-omics data integration techniques and advanced clustering algorithms to unravel the heterogeneity of diseases and uncover hidden subtypes within patient populations \cite{gliozzo2022heterogeneous}. One of the key challenges in this endeavor is the integration and analysis of diverse data modalities to uncover meaningful patterns and associations \cite{shen2009integrative, wang2014similarity}.

Traditional approaches often involve the independent analysis of each data type, or using manual integration, which may overlook important dependencies present in the data \cite{wang2014similarity, gliozzo2022heterogeneous}. To address this challenge, novel computational methods have emerged, aiming to integrate multi-omics data and extract insights for precision medicine applications \cite{nicora2020integrated}. These methods leverage advanced statistical and machine learning techniques to model the complex relationships between different omics and uncover hidden structures within the data \cite{gliozzo2022heterogeneous}.

In this study, we aim to address these challenges by employing state-of-the-art data integration techniques and clustering algorithms to explore disease subtype discovery using multi-omics data. Section~\ref{sec:methods} will describe our methodology for data preprocessing, integration, and clustering, as well as the metrics used to evaluate the performance of our approach. Furthermore, in the section~\ref{sec:results} we will present the results of our analysis and discuss the implications of our findings. Lastly, in section~\ref{sec:discussion}, we discuss the potential shortages of our analysis, as well as highlight avenues for future research in the field of multi-omics data analysis and precision medicine.
